\documentclass[12pt,a4paper]{article}
\usepackage[utf8]{inputenc}
\usepackage{amsmath}
\usepackage{amsfonts}
\usepackage{amssymb}
\usepackage{graphicx}
\usepackage[left=2cm,right=2cm,top=2cm,bottom=2cm]{geometry}

\begin{document}
\par\noindent
HINWEIS:
Zum erzeugen der Diagramme wurde ein Bash Script geschrieben, welches der Abgabe beiliegt. Dieses war nötig, da gnuplot Probleme beim erzeugen vieler Diagramme auf einmal hat, da dieses vom Terminal system abhängig ist. Daher erzeugt das Script buildPlot.sh alle png als auch gnuplot Dateien. Die gnuplot Dateien können aber wie gewohnt benutzt werden. 
\begin{figure}[h!]
\centering
\includegraphics[width=0.8\textwidth]{{WEAK_SCALING_JA.dat}.png}
\end{figure}
\begin{figure}[h!]
\centering
\begin{tabular}{c | c | c | c}
Prozesse & Knoten & Interlines & Zeit in Sekunden \\
1 & 1 & 100 & 37.2353 \\
2 & 1 & 141 & 36.7564 \\
4 & 2 & 200 & 37.3686 \\
8 & 4 & 282 & 39.5706 \\
16 & 4 & 400 & 40.0696 \\
24 & 4 & 490 & 40.0879 \\
64 & 8 & 800 & 39.2513 \\
\end{tabular}
\end{figure}
\par\noindent
Der Graph verhält sich sehr merkwürdig, da der Speedup weder Weak skaliert sondern bei sehr wenigen Prozessen sogar Strong. Die könnte sich erklären lassen, damit, dass die Kommunikation bei über mehrere Rechner verläuft und damit die Daten erst hin und her gesendet werden müssen. 
Dies wird ebenfalls unterstützt dadurch, dass pro Mainboard 2 CPUs verbaut sind und somit die Rechnungen mit zwei Knoten noch auf dem gleichen Mailboard ausgeführt werden. 
\newpage
\begin{figure}[h!]
\centering
\includegraphics[width=0.8\textwidth]{{WEAK_SCALING_GS.dat}.png}
\end{figure}

\begin{figure}[h!]
\centering
\begin{tabular}{c | c | c | c}
Prozesse & Knoten & Interlines & Zeit in Sekunden \\
1 & 1 & 100 & 36.7030\\
2 & 1 & 141 & 71.8596\\
4 & 2 & 200 & 73.0463\\
8 & 4 & 282 & 82.2029\\
16 & 4 & 400 & 76.5996\\
24 & 4 & 490 & 77.1833\\
64 & 8 & 800 & 77.6653\\
\end{tabular}
\end{figure}
\par\noindent
Dieser Verlauf verhält sich nicht äquivalent zu dem Jacobi Verfahren. Allerdings existiert bei beim Jakobi Verfahren keine Pipeline, somit kann es sein, dass hier die Laufzeit deutlich durch das Pipelining verzögert wird.
\newpage
\begin{figure}[h!]
\centering
\includegraphics[width=0.8\textwidth]{{STRONG_SCALING_JA.dat}.png}
\end{figure}
\begin{figure}[h!]
\centering
\begin{tabular}{c | c | c | c}
Prozesse & Knoten & Interlines & Zeit in Sekunden \\
12 & 1 & 960 & 186.6984\\
24 & 2 & 960 & 94.7785\\
48 & 4 & 960 & 49.3370\\
96 & 8 & 960 & 25.8823\\
120 & 10 & 960 & 21.7832\\
240 & 10 & 960 & 16.8659\\
\end{tabular}
\end{figure}
\par\noindent
Bei diesen Messungen tritt wie erwartet das Strong-Scaling auf, allerdings ist auch zu sehen, dass ab einer bestimmten Prozesszahl kaum noch Speedup auftritt. Dies kann sich damit erklären, dass die Kommunikation zwischen den Knoten zu lange dauert, als dass es noch zu einem Speedup kommt.
\newpage
\begin{figure}[h!]
\centering
\includegraphics[width=0.8\textwidth]{{STRONG_SCALING_GS.dat}.png}
\end{figure}
\begin{figure}[h!]
\centering
\begin{tabular}{c | c | c | c}
Prozesse & Knoten & Interlines & Zeit in Sekunden \\
12 & 1 & 960 & 371.0067\\
24 & 2 & 960 & 190.9016\\
48 & 4 & 960 & 96.5915\\
96 & 8 & 960 & 51.3105\\
120 & 10 & 960 & 42.3742\\
240 & 10 & 960 & 31.5057\\
\end{tabular}
\end{figure}
\par\noindent
Auch hier tritt das Strong-Scaling auf, allerdings ist hier wie schon bei Weak-Scaling zu sehen, dass das Gauss Seidel Verfahren deutlich langsamer ist, als das Jacobi Verfahren, was ebenfalls auf eine erhöhte kommunikation zurück führen könnte.
\newpage
\begin{figure}[h!]
\centering
\includegraphics[width=0.8\textwidth]{{COMMUNICATION_A_JA.dat}.png}
\end{figure}
\begin{figure}[h!]
\centering
\begin{tabular}{c | c | c | c}
Prozesse & Knoten & Interlines & Zeit in Sekunden \\
10 & 1 & 200 & 19.8375\\
10 & 2 & 200 & 19.8375\\
10 & 3 & 200 & 22.9522\\
10 & 4 & 200 & 27.3824\\
10 & 6 & 200 & 29.9228\\
10 & 8 & 200 & 30.7260\\
10 & 10 & 200 & 73.8324\\
\end{tabular}
\end{figure}

\begin{figure}[h!]
\centering
\includegraphics[width=0.8\textwidth]{{COMMUNICATION_A_GS.dat}.png}
\end{figure}
\newpage
\begin{figure}[h!]
\centering
\begin{tabular}{c | c | c | c}
Prozesse & Knoten & Interlines & Zeit in Sekunden \\
10 & 1 & 200 & 44.9182\\
10 & 2 & 200 & 48.7818\\
10 & 3 & 200 & 54.1011\\
10 & 4 & 200 & 55.8813\\
10 & 6 & 200 & 55.6212\\
10 & 8 & 200 & 57.5855\\
10 & 10 & 200 & 115.6501\\
\end{tabular}
\end{figure}
\par\noindent
Auf den letzten beiden Diagrammen ist der Tradeoff zwischen der Anzahl der Knoten und den Prozessen zu sehen. Hierbei ist deutlich zu erkennen, dass der Speedup deutlich zu nimmt je mehr sich die Anzahl der Prozesse den der Knoten nähern.
Dies folgt daraus, dass die Daten besser auf geteilt werden je mehr Prozesse existieren und es somit zu einem Speedup kommt.
Auch hier ist zu erkennen, dass das Gauss Seidel verfahren deutlich langsamer ist als das Jacobi Verfahren.
\end{document}